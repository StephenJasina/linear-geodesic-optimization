\subsection{Inputs and Outputs}
As input to the optimization process, we take an weighted undirected graph \(G = \pof{V_G, E_G}\). Each vertex \(s \in V_G\) represents a node in a network and is annotated with location information. We therefore can compute quantities like \(\GCL\pof{s, s'}\), the Great Circle Latency between \(s\) and \(s'\).

Additionally, an edge \(\cof{s, s'} \in E_G\) has an associated measured Round Trip Time \(\RTT\pof{s, s'}\). In practice, latencies are collected for almost every pair of nodes, so \(G\) is nearly complete.

We also take several hyperparameters. First is the residual latency threshold \(\epsilon\). Next are the \(\lambda\)'s, which are weighting parameters for each component of the loss function. Finally, there are a few hyperparameters describing the structure of the mesh.

We return a triangle mesh \(M = \pof{V_M, E_M}\), stored in a doubly connected edge list format. That is, each edge is actually stored as a pair of directed edges, except for on the boundary, where only a single directed edge is used. These directed edges trace out each face of the mesh counterclockwise.

As hinted at above, the actual vertex-edge connectivity of the mesh is to be selected before running the optimization. That said, each vertex has coordinates in \(\mathbb{R}^3\) which are to be chosen by the optimization algorithm. For the purposes of the algorithm and mesh regularity, we parameterize each vertex position with a single number. In our current implementation, each vertex \(v \in V_M\) can be broken into parts as \(\pof{p_v, z_v}\), where \(p_v\) is a latitude-longitude pair, and \(z_v\) is an altitude. The optimization algorithm then determines the best \(z\) values.

Note that the above parameterization implies we can map vertices \(s \in V_G\) to positions on the mesh \(\pi\pof{s}\). For the purposes of the following computations, assume the stronger statement \(\pi\pof{s} \in V_M\). While this assumption is not strictly necessary, it significantly simplifies the geodesic distance computation. Furthermore, provided our mesh is fine enough, the stronger assumption will lead to minimal numerical error.
