Here are the variables used for this part of the computation: \begin{center}\begin{tabular}{r|l}
	\(\gamma\) & Set of points in \(V_M\) \\ \hline
	\(h\) & Mean half-edge length \\ \hline
	\(\delta^\gamma\) & Heat source (indicator on \(\gamma\)) \\ \hline
	\(u^{\gamma, \text{N}}\) & Heat flow with zero-Neumann boundary condition \\ \hline
	\(u^{\gamma, \text{D}}\) & Heat flow with zero-Dirichlet boundary condition \\ \hline
	\(u^\gamma\) & Heat flow \\ \hline
	\(q^\gamma_{i, j}\) & Intermediate value for computation \\ \hline
	\(m^\gamma_{i, j}\) & Intermediate value for computation \\ \hline
	\(X^\gamma_{i, j}\) & Unit vector in same direction as \(\nabla u^\gamma_{i, j}\) \\ \hline
	\(p^\gamma_{i, j}\) & Intermediate value for computation \\ \hline
	\(\phi^\gamma\) & Vector of geodesic distances
\end{tabular}\end{center}

\subsubsection{Forward Computation}

Say we want to find the geodesic distances to a set of points \(\gamma \subseteq V_M\). Following the \href{https://www.cs.cmu.edu/~kmcrane/Projects/HeatMethod/}{Crane et al's Heat Method}, we use the (approximate) heat flow \(u^\gamma\), where

\begin{align*}
	h &= \text{TODO}, \\
	\delta^\gamma &= \begin{cases}
		1 & \text{if \(v_i \in \gamma\)}, \\
		0 & \text{if \(v_i \not\in \gamma\)},
	\end{cases} \\
	u^{\gamma, \text{N}} &= \pof{D - h^2L_C^{\text{N}}}^{-1}\delta^\gamma, \\
	u^{\gamma, \text{D}} &= \pof{D - h^2L_C^{\text{D}}}^{-1}\delta^\gamma, \\
	u^\gamma &= \frac{1}{2}\pof{u^{\gamma, \text{N}} + u^{\gamma, \text{D}}}, \\
	q^\gamma_{i, j} &= u^\gamma_i\pof{v_{c\pof{i, j}} - v_j}, \\
	m^\gamma_{i, j} &= q^\gamma_{i, j} + q^\gamma_{j, c\pof{i, j}} + q^\gamma_{c\pof{i, j}, i}, \\
	\pof{\nabla u^\gamma}_{i, j} &= N_{i, j} \times m^\gamma_{i, j}, \\
	X^\gamma_{i, j} &= -\frac{\pof{\nabla u^\gamma}_{i, j}}{\norm{\pof{\nabla u^\gamma}_{i, j}}_2}, \\
	p_{i, j} &= \cot\pof{\theta_{i, j}}\pof{v_j - v_i}, \\
	\pof{\nabla \cdot X^\gamma}_i &= \frac{1}{2}\sum_{\substack{k \\ \text{\(\pof{v_i, v_k}\) is} \\ \text{a half-edge}}}\pof{p_{i, k} - p_{c\pof{i, k}, i}} \cdot X^\gamma_{i, k}, \\
	\phi^\gamma &= \pof{L_C^{\text{N}}}^+ \cdot \pof{\nabla \cdot X^\gamma}.
\end{align*}

Here, \(\pof{L_C^{\text{N}}}^+\) is the \href{https://en.wikipedia.org/wiki/Moore%E2%80%93Penrose_inverse}{pseudoinverse} of \(L_C^{\text{N}}\) (as it is singular).

Note that we're being careful about which pieces have a dependence on \(\gamma\), as we can reuse certain computations if we want to compute distances from multiple sources. We can get the distance matrix (that is, get rid of the \(\gamma\) dependence) from \[\phi_{i, j} = \pof{\phi^{\cof{v_j}}}_i.\]

\subsubsection{Reverse Computation}

Note that \(c\pof{i, c\pof{j, i}} = j\). This is helpful for reindexing some sums (in particular, the one for \(\nabla \cdot X\)).

We then have the following partial derivatives:

\begin{align*}
	\frac{\partial h}{\partial \rho_\ell} &= \text{TODO} \\
	\frac{\partial u^{\gamma, \text{N}}}{\partial \rho_\ell} &= -\pof{D - h^2L_C^{\text{N}}}^{-1}\pof{\frac{\partial D}{\partial \rho_\ell} - 2h\frac{\partial h}{\partial \rho_\ell}L_C^{\text{N}} - h^2\frac{\partial L_C}{\partial \rho_\ell}}u^{\gamma, \text{N}}, \\
	\frac{\partial u^{\gamma, \text{D}}}{\partial \rho_\ell} &= -\pof{D - h^2L_C^{\text{D}}}^{-1}\pof{\frac{\partial D}{\partial \rho_\ell} - 2h\frac{\partial h}{\partial \rho_\ell}L_C^{\text{D}} - h^2\frac{\partial L_C}{\partial \rho_\ell}}u^{\gamma, \text{D}}, \\
	\frac{\partial u^\gamma}{\partial \rho_\ell} &= \frac{1}{2}\pof{\frac{\partial u^{\gamma, \text{N}}}{\partial \rho_\ell} + \frac{\partial u^{\gamma, \text{D}}}{\partial \rho_\ell}}, \\
	\frac{\partial q^\gamma_{i, j}}{\partial \rho_\ell} &= \begin{cases}
		\frac{\partial u^\gamma_i}{\rho_\ell}\pof{v_{c\pof{i, j}} - v_j} - u^\gamma_i\frac{\partial v_\ell}{\rho_\ell} & \text{if \(\ell = j\)}, \\
		\frac{\partial u^\gamma_i}{\rho_\ell}\pof{v_{c\pof{i, j}} - v_j} + u^\gamma_i\frac{\partial v_\ell}{\partial \rho_\ell} & \text{if \(\ell = c\pof{i, j}\)}, \\
		\frac{\partial u^\gamma_i}{\rho_\ell}\pof{v_{c\pof{i, j}} - v_j} & \text{otherwise},
	\end{cases} \\
	\frac{\partial m^\gamma_{i, j}}{\partial \rho_\ell} &= \frac{\partial q^\gamma_{i, j}}{\partial \rho_\ell} + \frac{\partial q^\gamma_{j, c\pof{i, j}}}{\partial \rho_\ell} + \frac{\partial q^\gamma_{c\pof{i, j}, i}}{\partial \rho_\ell}, \\
	\frac{\partial \pof{\nabla u^\gamma}_{i, j}}{\partial \rho_\ell} &= \frac{\partial N_{i, j}}{\partial \rho_\ell} \times m^\gamma_{i, j} + N_{i, j} \times \frac{\partial m^\gamma_{i, j}}{\partial \rho_\ell}, \\
	\frac{\partial X^\gamma_{i, j}}{\partial \rho_\ell} &= -\frac{1}{\norm{\pof{\nabla u^\gamma}_{i, j}}_2}\pof{I - X^\gamma_{i, j}\pof{X^\gamma_{i, j}}^\intercal}\frac{\partial \pof{\nabla u^\gamma}_{i, j}}{\partial \rho_\ell}, \\
	\frac{\partial p_{i, j}}{\partial \rho} &= \begin{cases}
		\pof{\frac{\partial}{\partial \rho_\ell}\cot\pof{\theta_{i, j}}}\pof{v_j - v_i} - \cot\pof{\theta_{i, j}}\frac{\partial v_\ell}{\rho_\ell} & \text{if \(\ell = i\)}, \\
		\pof{\frac{\partial}{\partial \rho_\ell}\cot\pof{\theta_{i, j}}}\pof{v_j - v_i} + \cot\pof{\theta_{i, j}}\frac{\partial v_\ell}{\rho_\ell} & \text{if \(\ell = j\)}, \\
		\pof{\frac{\partial}{\partial \rho_\ell}\cot\pof{\theta_{i, j}}}\pof{v_j - v_i} & \text{if \(\ell = c\pof{i, j}\)}, \\
		0 & \text{otherwise},
	\end{cases} \\
	\frac{\partial \pof{\nabla \cdot X^\gamma}_i}{\partial \rho_\ell} &= \frac{1}{2}\sum_{\substack{k \\ \text{\(\pof{v_i, v_k}\) is} \\ \text{a half-edge}}}\pof{\pof{\frac{\partial p_{i, k}}{\partial \rho_\ell} - \frac{\partial p_{c\pof{i, k}, i}}{\partial \rho_\ell}} \cdot X^\gamma_{i, k} + \pof{p_{i, k} - p_{c\pof{i, k}, i}} \cdot \frac{\partial X^\gamma_{i, k}}{\partial \rho_\ell}} \\
	\frac{\partial \phi^\gamma}{\partial \rho_\ell} &= \pof{L_C^{\text{N}}}^+\pof{\frac{\partial \pof{\nabla \cdot X^\gamma}}{\partial \rho_\ell} - \frac{\partial L_C^{\text{N}}}{\partial \rho_\ell}\phi^\gamma}.
\end{align*}
